% Chinese text
\usepackage{fontspec}
\usepackage{xeCJK}
% Set an English (Latin) font
\setmainfont{TeX Gyre Pagella}  % or another font you have installed
% Use a Chinese-capable font that shows up in your fc-list output
\setCJKmainfont{AR PL UMing CN}

% ---- add headings for #### and ##### ----

\usepackage{titlesec}

% Define formatting for \paragraph (level 4 heading)
\titleformat{\paragraph}
  {\normalfont\normalsize\bfseries}{\theparagraph}{1em}{}
\titlespacing*{\paragraph}
  {0pt}{3.25ex plus 1ex minus .2ex}{1.5ex plus .2ex}

% Define formatting for \subparagraph (level 5 heading)
\titleformat{\subparagraph}
  {\normalfont\normalsize\bfseries\itshape}{\thesubparagraph}{1em}{}
\titlespacing*{\subparagraph}
  {0pt}{3.25ex plus 1ex minus .2ex}{1.5ex plus .2ex}


% Center section headings
\titleformat{\section}
  {\normalfont\Large\bfseries\centering} % Format: centered, bold, large
  {\thesection} % Label (section number)
  {1em} % Space between number and title
  {} % Title text

% Center subsection headings (optional)
\titleformat{\subsection}
  {\normalfont\large\bfseries\centering} % Format: centered, bold, slightly smaller
  {\thesubsection} % Label (subsection number)
  {1em} % Space between number and title
  {} % Title text

% Adjust spacing if needed (optional)
\titlespacing*{\section}{0pt}{1.5em}{1em} % Left, before, after spacing
\titlespacing*{\subsection}{0pt}{1.5em}{1em}

% ---- more spacing for young learners ----

% Increase overall line spacing for a more open, readable layout
% \usepackage{setspace}

% ---- Custom styling for block quotes ----
\usepackage{mdframed}
\usepackage{xcolor}
% \usepackage{needspace}

\renewenvironment{quote}
  {%

    \begin{samepage}%
    \begin{mdframed}[
      backgroundcolor=gray!10,
      linecolor=gray!50,
      innerleftmargin=18pt,
      innerrightmargin=18pt,
      innertopmargin=15pt,
      innerbottommargin=12pt,
      skipabove=20pt,
      skipbelow=20pt
    ]%
    \begin{center}
  }
  {%
\end{center}
    \end{mdframed}%
    \end{samepage}%
  }


\newcommand{\myhrule}{%
  \par\vspace*{1em}%
  {\centering\rule{0.6\linewidth}{0.4pt}\par}%
  \vspace*{1em}%
}

% Plots
\usepackage{pgfplots}
\pgfplotsset{width=10cm,compat=1.9}
\usepgfplotslibrary{statistics}

\usepackage{fancyhdr}
\pagestyle{fancy}
\fancyhf{} % clear all header and footer fields
\fancyfoot[LE,RO]{\thepage} % left footer on even pages, right footer on odd pages
\renewcommand{\headrulewidth}{0pt}
\renewcommand{\footrulewidth}{0pt}
\usepackage{tikz}
\usepackage{pgfmath}
\usepackage{ifthen}

\newcommand{\hexagram}[6]{%
\begin{center}
  \begin{tikzpicture}[scale=0.8]
  \def\hexone{#1}%
  \def\hextwo{#2}%
  \def\hexthree{#3}%
  \def\hexfour{#4}%
  \def\hexfive{#5}%
  \def\hexsix{#6}%

    % Loop over line index (from top: 6 to bottom: 1) with associated value
    \foreach \i/\val in {6/\hexsix,5/\hexfive,4/\hexfour,3/\hexthree,2/\hextwo,1/\hexone} {
      \pgfmathparse{-(7-\i)*0.7}%
      \let\y\pgfmathresult
      \ifnum\val=1
        % Yang: Solid line
        \draw (0,\y) -- (4,\y);
      \else
        % Yin: Broken line (two segments)
        \draw (0,\y) -- (1.5,\y);
        \draw (2.5,\y) -- (4,\y);
      \fi
    }
  \end{tikzpicture}%
  \vspace{2em}
\end{center}
}

\newcommand{\consultation}[6]{%
\begin{center}
  \begin{tikzpicture}[scale=0.8]
    \def\hexone{#1}%
    \def\hextwo{#2}%
    \def\hexthree{#3}%
    \def\hexfour{#4}%
    \def\hexfive{#5}%
    \def\hexsix{#6}%
    % Loop from top (line 6) to bottom (line 1)
    \foreach \i/\val in {6/\hexsix,5/\hexfive,4/\hexfour,3/\hexthree,2/\hextwo,1/\hexone} {%
      \pgfmathparse{-(7-\i)*0.7}%
      \let\y\pgfmathresult
      \ifnum\val=7
         % Unchanging Yang: Solid line
         \draw (0,\y) -- (4,\y);
      \else\ifnum\val=8
         % Unchanging Yin: Broken line
         \draw (0,\y) -- (1.5,\y);
         \draw (2.5,\y) -- (4,\y);
      \else\ifnum\val=6
         % Changing Yin: Broken line with "O"
         \draw (0,\y) -- (1.5,\y);
         \draw (2.5,\y) -- (4,\y);
         \node at (2,\y) {\Large O};
      \else\ifnum\val=9
         % Changing Yang: Solid line with "X"
         \draw (0,\y) -- (4,\y);
         \node at (2,\y) {\Large X};
      \fi\fi\fi\fi
    }
  \end{tikzpicture}%
  \vspace{2em}
\end{center}
}


\usepackage{enumitem}
% \setlist{itemsep=1ex, topsep=1ex}
% \setlist[itemize]{leftmargin=10pt, itemsep=10pt}
\setlist[itemize]{leftmargin=10pt, itemsep=10pt, topsep=10pt, parsep=10pt}
